\subsection{Privilege Escalation via Deja Vu}
Der Benutzer \texttt{dev} hat Zugriff auf das Verzeichnis \texttt{/opt/dejavu}, aus dem bereits im Screenshot von Agent Smith eine Datei geöffnet war.

In dem Verzeichnis findet sich ein Interpreter für die Deja Vu Programmiersprache, welche die Maschinen zur Programmierung von künstlicher Intelligenz einsetzen -- zurzeit jedoch nur in Katzenprogrammen, wie in der beiliegenden \texttt{README.md} angemerkt wird.

\begin{lstlisting}
dr-xr-x---    1 dev      dev           4096 Jul  4 10:22 .
drwxr-xr-x    1 root     root          4096 Jul  4 10:22 ..
-r--r-----    1 dev      dev           1657 Jun 30 09:20 README.md
-r-xr-x---    1 dev      dev        2034928 Jul  4 10:21 dejavu
dr-xr-x---    1 dev      dev           4096 Jul  4 10:22 examples
-r--r-----    1 dev      dev           4980 Jul  3 09:58 main.go
\end{lstlisting}

Ferner enthält die \texttt{README.md} eine kurze Übersicht der erlaubten Instruktionen sowie einen Verweis auf Beispiel-Skripts im \texttt{examples}-Verzeichnis.
Bei Deja Vu handelt es sich um eine Abwandlung der esoterischen Programmiersprache Brainfuck bzw. deren Variante \href{https://de.wikipedia.org/wiki/Ook!}{Ook!}

Nach einem kurzen Blick in den ebenfalls beiliegenden Quellcode kann festgestellt werden, dass eine weitere, undokumentierte Instruktion unterstützt wird, welche das Ausführen externer Programme ermöglicht; jedoch erschließt sich ein möglicher Angriff erstmal nicht, denn der Interpreter ist keine setuid-Binary, wird also ganz normal als Benutzer \texttt{dev} ausgeführt.
Jedoch findet man nach ausführlicher Untersuchung heraus, dass die Interpreter-Binary die Linux~Capability \texttt{CAP\_SETFCAP} über die erweiterten Dateisystem-Attribute gesetzt hat, darauf gab es bereits in der TODO-Liste im Screenshot einen Hinweis.

\begin{lstlisting}
$ getcap dejavu
dejavu cap_setfcap=eip
\end{lstlisting}

Die \texttt{CAP\_SETFCAP}-Capability erlaubt dem ausgeführten Interpreter-Prozess beliebiges Setzen weiterer Capabilities im Dateisystem.
Mehrere Wege führen hier zum Ziel, in diesem Walkthrough wird die \texttt{CAP\_DAC\_OVERRIDE}-Capability der \texttt{/bin/busybox} zugewiesen, wodurch für alle Busybox-Tools wie \texttt{ls} oder \texttt{cat} die Rechteüberprüfung auf Dateisystemebene deaktiviert wird.

Normalerweise würde ein Administrator dazu folgenden Befehl ausführen:
\begin{lstlisting}
$ /usr/sbin/setcap cap_dac_override+eip /bin/busybox
unable to set CAP_SETFCAP effective capability: Operation not permitted
\end{lstlisting}

Mit dem passenden DejaVu-Skript sollte dieser Befehl aber erfolgreich ausgeführt werden.
Dieses Skript wird am besten automatisiert generiert, denn die undokumentierte Instruktion (\texttt{Deja? Vu?}) zum Ausführen eines Programms erfordert, dass die Zelleninhalte relativ zum Zeiger in einer bestimmten Struktur vorliegen.
Demnach steht in der ersten Zelle die Länge des ersten Zeichenkette, gefolgt von dieser Zeichenkette (pro Zelle ein Byte).
In der nächsten Zelle steht dann die Länge der zweiten Zeichenkette, gefolgt von der zweiten Zeichenkette.
Die Argumentliste terminiert bei einer 0.

\begin{figure}[!ht]
\centering
\begin{tikzpicture}
\matrix[
    matrix of nodes,
    every node/.style={
        draw,
        anchor=center,
        minimum height=20pt,
        minimum width=20pt,
        text=blue
    }
]{
    \strut \color{red}9 & \strut / & \strut b & \strut i & \strut n & \strut / & \strut e & \strut c & \strut h & \strut o &
    \strut \color{red}5 & \strut H & \strut a & \strut l & \strut l & \strut o &
    \strut \color{red}0 \\
};
\end{tikzpicture}
\caption{Struktur der Zellen mit \texttt{/bin/echo Hallo} als Beispiel}
\end{figure}

Der folgende JavaScript-Code generiert ein DejaVu-Skript, welches den gewünschten \texttt{setcap}-Befehl korrekt in die Zellen organisiert und die undokumentierte Instruktion ausführt:

\lstinputlisting[language=JavaScript]{dejavu/buildDejavu.js}

Es wird ein ca.~50000-Zeichen langes DejaVu-Skript generiert, welches anschließend auf die Maschine als \texttt{/tmp/exploit.dv} kopiert und mit dem Interpreter ausgeführt werden kann:

\begin{lstlisting}
$ whoami
dev
$ ls /root
ls: can't open '/root': Permission denied
$ ./dejavu /tmp/exploit.dv
$ whoami
dev
$ ls /root
token.txt
$ cat /root/token.txt
HC2021{WhoEvenNeedsRootWhenThereAreCaps}
\end{lstlisting}
