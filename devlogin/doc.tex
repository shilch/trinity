\subsection{Exploit des \texttt{devlogin}}
Bei Agent Smith handelt es sich noch nicht um den root-Benutzer, es ist also eine weitere Privilegienerhöhung notwendig.
Nach kurzer Untersuchung des Dateisystems kann das Verzeichnis \texttt{/opt/devlogin} identifiziert werden, welches speziell der Gruppe der \texttt{agents}, die Gruppe des eingeloggten Benutzers, gehört.

Die Berechtigungen im Verzeichnis sind wie folgt gesetzt:
\begin{lstlisting}
dr-xr-x---    1 dev      agents        4096 Jul  1 23:48 .
drwxr-xr-x    1 root     root          4096 Jul  1 23:48 ..
-r-sr-x---    1 dev      agents       28008 Jul  1 23:33 devlogin
-r--r-----    1 dev      agents        1126 Jul  1 23:33 devlogin.cpp
\end{lstlisting}

Es liegt die setuid-Binary \texttt{devlogin} vor, welche bei der Ausführung die effektive UID auf die des Nutzers \texttt{dev} setzt.
Das Programm fragt nach Ausführung nach einem Benutzernamen und einem Passwort.
Ein kurzer Blick in den zugehörigen Quellcode verrät, dass der \texttt{devlogin} eine \texttt{dev}-Shell gibt, wenn die Zugangsdaten korrekt sind:
\lstinputlisting[linerange={36-44},firstnumber=36]{devlogin/devlogin.cpp}

Zudem kann aus dem Quellcode der beim Login gewünschte Benutzername (ebenfalls \texttt{dev}) sowie der MD5-Hash des Passworts (\texttt{14754f13e5280c5d49d2ae536c2d57e2}) entnommen werden.
Es lässt sich schnell ermitteln, dass das Passwort \texttt{machine} lautet.

Jedoch stellt man bei Eingabe dieser Zugangsdaten fest, dass immer noch keine Shell erzeugt wurde; das Problem:
Die Funktion, die zum hexadezimalen Kodieren des Passworthashes verwendet wurde, erzeugt einen Hexstring in Großbuchstaben und mit einem Doppelpunkt als Trennzeichen zwischen den Oktetten.
Der Zielhash hingegen ist in Kleinbuchstaben ohne Trennzeichen hinterlegt.
Es gibt also keine Möglichkeit, dass die Hexstrings identisch sein können.
Für eine Privilegienerhöhung muss also ein anderer Weg gefunden werden.

Im Rahmen der Challenge ist ein Exploit via \texttt{LD\_PRELOAD}-Umgebungsvariable vorgesehen, die den dynamischen Linker dazu bringt, eine benutzerdefinierte Bibliothek zu laden.
Zufälligerweise ist im Docker-Image bereits ein C-Compiler vorinstalliert, sodass die Bibliothek direkt im Container selbst erzeugt werden kann.
Dazu wird eine C-Datei \texttt{inject.c} mit folgendem Inhalt angelegt (\hbox{z. B.} im \texttt{/tmp}-Verzeichnis mit \texttt{vi}):
\lstinputlisting[language=C,linerange={1-8}]{devlogin/ld_preload.txt}

Die \texttt{\_init}-Funktion wird aufgerufen, sobald die dynamische Library geladen wurde, und ersetzt den aktuellen Prozess durch eine Shell.
Das \texttt{unsetenv} stellt lediglich sicher, dass die Bibliothek nicht in einer Endlosschleife sich selbst lädt.
Anschließend wird der Code mit folgendem Befehl in eine dynamische Bibliothek kompiliert:
\lstinputlisting[linerange={10}]{devlogin/ld_preload.txt}

Die Privilegienerhöhung erfolgt anschließend durch setzen von \texttt{LD\_PRELOAD} auf die kompilierte Bibliothek vor der Ausführung des \texttt{devlogin}-Programms:
\begin{lstlisting}
$ id
uid=1000(smith) gid=1000(agents) groups=27(video),1000(agents),1000(agents)
$ LD_PRELOAD=/tmp/inject.so ./devlogin
$ id
uid=1002(dev) gid=1000(agents) groups=27(video),1000(agents),1000(agents)
\end{lstlisting}

Mit Shell-Zugriff zum \texttt{dev}-Benutzer kann auch das nächste Token unter \texttt{/home/dev/token.txt} gelesen werden:
\lstinputlisting{tokens/token3}

Nun ist es verwunderlich, dass \texttt{LD\_PRELOAD} auf einer setuid-Binary funktioniert, wie das Token auch andeutet.
Tatsächlich wird die eigene Bibliothek nicht in den \texttt{devlogin} injiziert, sondern in den Kindprozess, der in Zeile~18 erzeugt wird:
\lstinputlisting[language=C,linerange={13-19},firstnumber=13]{devlogin/devlogin.cpp}
Das Docker-Image ist ein musl-basiertes System, d.h. alle Binaries benutzen den dynamischen Musl-Linker.
Dies kann über den \texttt{file}-Befehl auf der Binary verifiziert werden.
Anders als der GNU dynamische Linker\footnote{\url{https://github.com/bminor/glibc/blob/master/sysdeps/generic/unsecvars.h\#L22}} löscht musl \textbf{nicht} die \texttt{LD\_PRELOAD}-Umgebungsvariable bei setuid-Binaries, sondern ignoriert diese lediglich.
Sie wird also trotzdem an Kindprozesse weitergegeben.
Voraussetzung ist nur, dass die echte UID mit der effektiven UID übereinstimmt\footnote{\url{https://github.com/esmil/musl/blob/master/src/ldso/dynlink.c\#L1121-L1123}}, dafür sorgt Zeile 14.
Eine korrekte Implementierung würde den Code in Zeile~14 erst nach erfolgreichem Login in Zeile~42 durchführen.
